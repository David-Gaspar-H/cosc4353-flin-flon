%! Author = davidranamagar
%! Date = 3/18/25

\documentclass[11pt]{article}
\usepackage[margin=1in]{geometry}
\usepackage{amsmath}
\usepackage{amssymb}
\usepackage{graphicx}
\usepackage{enumitem}
\usepackage{array}
\usepackage{booktabs}
\usepackage{multirow}
\usepackage{tikz}
\usepackage{hyperref}

% Custom checkbox command
\newcommand{\checkbox}{\tikz\draw[scale=0.5,thick] (0,0) rectangle (1,1);}

\begin{document}

\begin{center}
    {\Large \textbf{Reduced Course Load (RCL) Form for Undergraduates}}

    \vspace{0.2cm}
    Phone: (713) 743-5065 \quad Email: \href{mailto:isssohlp@central.uh.edu}{isssohlp@central.uh.edu} \quad \href{http://uh.edu/oisss}{http://uh.edu/oisss}
\end{center}

\noindent F-1 students are required to maintain full-time enrollment while studying in the U.S. Undergraduate and post-baccalaureate students are expected to complete a minimum of 12 hours of course work during the fall and spring semesters. Classes during the summer are optional unless it is the first semester at UH; then an F-1 student has to complete 6 hours (i.e. full-time for summer). The following form must be completed before dropping below full-time hours after start of classes.

\noindent \textbf{Note:} Dropping below full course load may involve the loss of resident tuition based on a scholarship, grant, or on-campus employment.

\vspace{0.5cm}
\noindent \textbf{Please complete the form below by selecting one of the options:}

\vspace{0.3cm}
\noindent \textbf{1. ACADEMIC DIFFICULTY (FIRST SEMESTER ONLY)}\\
RCL for valid academic difficulties is allowed once and only in the first semester when starting a new degree program. A minimum of 6hrs will still have to completed. This option cannot be used or submitted prior to ORD.

\begin{itemize}[leftmargin=0.5cm]
    \item[\checkbox] \textbf{Initial Adjustment Issues (IAI)}\\
    I am having initial difficulties with the English language, reading requirements, or unfamiliarity with American teaching methods.\\
    Please explain: \underline{\hspace{10cm}}

    \vspace{0.3cm}
    \item[\checkbox] \textbf{Improper Course Level Placement (ICLP)}\\
    I am having difficulty with my class(es) due to improper course level placement which may include not having the prerequisites or insufficient background to complete the course at this time. For example, an international student taking U.S. History for the first time (e.g. no previous exposure, insufficient background) or a philosophy course that is based on a worldview that clashes with the student's own culture.
\end{itemize}

\noindent \textbf{ICLP CERTIFYING SIGNATURE BY PROFESSOR}\\
I recommend that this student be allowed to drop the following course(s) due to improper course level placement as defined above.

\begin{tabular}{llll}
Class\underline{\hspace{1.5cm}} & Professor \underline{\hspace{1.5cm}} & Signature\underline{\hspace{2cm}} & Date\underline{\hspace{1.5cm}} \\
Class\underline{\hspace{1.5cm}} & Professor \underline{\hspace{1.5cm}} & Signature\underline{\hspace{2cm}} & Date\underline{\hspace{1.5cm}} \\
Class\underline{\hspace{1.5cm}} & Professor \underline{\hspace{1.5cm}} & Signature\underline{\hspace{2cm}} & Date\underline{\hspace{1.5cm}} \\
\end{tabular}

\vspace{0.5cm}
\noindent \textbf{2. MEDICAL REASON}
\begin{itemize}[leftmargin=0.5cm]
    \item[\checkbox] Valid medical reason must be proven with a supporting letter from a licensed medical doctor, clinical psychologist, or doctor of osteopathy. The letter has to contain the following information: written in English on a letterhead, signed in ink, the recommended credit hours of enrollment, when the below hours should begin and end (if known), details of when student first saw the doctor, and when they advised the student to withdraw from course(s). Medical excuses must be renewed each semester. You are only allowed to accumulate 12 months of reduced course load for medical reasons during any given degree level. Zero hours are allowed under this provision of the law only if it is clearly recommended by the licensed medical professional.

    \vspace{0.3cm}
    \item[\checkbox] Letter from a licensed medical doctor, doctor of osteopathy, a licensed psychologist/clinical psychologist is attached.
\end{itemize}

\vspace{0.5cm}
\noindent \textbf{3. FINAL SEMESTER}
\begin{itemize}[leftmargin=0.5cm]
    \item[\checkbox] This is my final semester and I only need \underline{\hspace{1cm}} hours of course work to complete my degree. I understand that if I am granted a reduced course load and fail to complete my degree as planned, I may be in violation of my legal status and will need to apply for reinstatement. (If you need only one course to finish your program of study, it cannot be taken through online/distance education).
\end{itemize}

\vspace{0.5cm}
\noindent \textbf{4. CONCURRENTLY ENROLLED}
\begin{itemize}[leftmargin=0.5cm]
    \item[\checkbox] I am taking courses at another college/University and want to drop a course at UH. I will still have 12 hours of enrollment between both schools. After the drop, I will have \underline{\hspace{1cm}} hours at UH and \underline{\hspace{1cm}} hours at \underline{\hspace{2cm}} (school name). Attach proof of concurrent enrollment. Academic advisor signature is not required for this option, only ISSSO counselor.
\end{itemize}

\vspace{0.5cm}
\noindent I am applying for a reduced course load for the \checkbox~fall semester of 20\underline{\hspace{1cm}} \checkbox~spring semester of 20\underline{\hspace{1cm}}

\noindent I want to drop the following class(es): \underline{\hspace{1.5cm}}; \underline{\hspace{1.5cm}}; \underline{\hspace{1.5cm}} (course number). After the drop, I will have a total of \underline{\hspace{1cm}} hours (at UH) for the: \checkbox~Fall semester 20\underline{\hspace{1cm}}. \checkbox~Spring semester of 20\underline{\hspace{1cm}}.

\vspace{0.5cm}
\noindent You must submit a copy of this form to Office of the University Registrar (located in the Welcome Center) if you are requesting the drop after the 1\textsuperscript{st} day of the semester. The approval signature from your Academic Advisor and ISSSO are required to drop a course. You may still be responsible for the tuition and fee charges to the dropped course(s) after passing the deadline.

\vspace{0.5cm}
\noindent Your Name: \underline{\hspace{3cm}} Signature: \underline{\hspace{3cm}} PS ID: \underline{\hspace{2cm}} Date: \underline{\hspace{2cm}}

\vspace{0.5cm}
\noindent \textbf{APPROVAL SIGNATURE FROM ACADEMIC ADVISOR}\\
Name: \underline{\hspace{3cm}} Signature: \underline{\hspace{3cm}} Date: \underline{\hspace{2cm}}

\vspace{0.5cm}
\noindent \textbf{APPROVAL SIGNATURE FROM ISSSO (if course drop is required)}\\
Name: \underline{\hspace{3cm}} Signature: \underline{\hspace{3cm}} Date: \underline{\hspace{2cm}}

\end{document}

